
% resumo em português
\setlength{\absparsep}{12pt} % ajusta o espaçamento dos parágrafos do resumo
\begin{resumo}

Ao agendar avaliações semestrais de uma universidade, pode-se identificar um clássico problema de combinatória, que vem sendo objeto de estudo na literatura desde a década de 1960. Este trabalho objetiva-se em apresentar o problema de programação de horários de exames universitários, suas características e restrições, bem como a modelagem matemática das turmas e alunos com a utilização de Grafos de Interseção. Será proposta uma heurística para aplicar o método de solução recorrendo à otimização de um problema de Coloração Generalizada de Grafos. Neste trabalho será apresentado um protótipo desenvolvido, e sua respectiva documentação, para permitir aos usuários a utilização do método de solução proposto de forma amigável.

 \textbf{Palavras-chave}: Problema de Programação de Horários de Exames em Universidades. Grafos de Interseção. Coloração de Grafos. Otimização.

\end{resumo}

% resumo em inglês
\begin{resumo}[Abstract]
 \begin{otherlanguage*}{english}
 
When scheduling semester examinations at a university, one can identify a classic combinatorial problem, which has been a study in the literature since the 1960s. This work aims to present the Examination Timetabling Problem, their characteristics and restrictions, as well as the mathematical modeling of classes with the use of Intersection Graphs. A heuristic will be proposed to apply the solution method using the optimization of a Generalized Graph Coloring Problem. In this work, will be presented a developed prototype and its respective documentation to allow users to use the proposed solution method in a friendly way.

   \vspace{\onelineskip}

\noindent
  \textbf{Keywords}:  Examination Timetabling Problem. Intersection Graphs. Graph Coloring. Optimization.
 \end{otherlanguage*}
\end{resumo}

% resumo em francês
%\begin{resumo}[Résumé]
% \begin{otherlanguage*}{french}
%    Il s'agit d'un résumé en français.

%   \textbf{Mots-clés}: latex. abntex. publication de textes.
% \end{otherlanguage*}
%\end{resumo}

% resumo em espanhol
%\begin{resumo}[Resumen]
% \begin{otherlanguage*}{spanish}
%   Este es el resumen en español.
%
%   \textbf{Palabras clave}: latex. abntex. publicación de textos.
% \end{otherlanguage*}
%\end{resumo}
% ---