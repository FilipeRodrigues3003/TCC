\begin{apendicesenv}

% Imprime uma página indicando o início dos apêndices
\partapendices

% ----------------------------------------------------------
\chapter{Roteiro}
\label{cap:roteiro}
A seguir está o roteiro para realização do teste de usabilidade do sistema desenvolvido, e de forma conjunta as perguntas referentes ao questionário de percepção do usuário.

\begin{enumerate}
    \item[I --] \textbf{Apresentação}

Agradeço a participação no teste de usabilidade do protótipo do sistema de agendamento de provas.

    \item[II --] \textbf{Identificação do Usuário}

Antes de iniciarmos é necessário informar os seguintes dados:

\begin{itemize}
    \item Qual seu nome?
    \item Qual seu “e-mail”?
    \item Qual profissão ou ocupação?
    \item Já utilizou ou viu esse sistema anteriormente?
    \item Conhece o trabalho ao qual o sistema pertence?
\end{itemize} 

Esses dados serão utilizados para permitir quantificar sua experiência com a usabilidade do sistema.

    \item[III --] \textbf{Identificação do Sistema}
   
Você estará sendo orientado(a), conforme as ação você deverá tomar ao utilizar o sistema, e simultaneamente serão realizadas perguntas sobre sua experiência. 
 
\begin{citacao}
    Neste momento do teste o usuário deverá ser conduzido até á página onde se encontra o sistema, e deverá ser realizado login utilizando um usuário predefinido que só é possuidor do espaço de trabalho padrão: “exemplo”.
\end{citacao}

Observe a página atual do sistema por 15 segundos e responda:

\begin{itemize}
    \item Qual a finalidade dessa página para você? 
    \item Você consegue dizer quais os recursos estão disponíveis nessa página?
\end{itemize}

Agora identifique onde está o menu principal do sistema e visualize suas informações.

Agora crie um novo \textbf{Espaço de Trabalho} com o título “Minha Escola”.

Acesse o \textbf{Espaço de Trabalho} “Minha Escola”.

Retorne até a página “\textbf{Dashboard}”, por favor.

Apague o \textbf{Espaço de Trabalho} “Minha Escola”. 

Responda a seguir:

\begin{itemize}
    \item Você teve alguma dificuldade em seguir alguma instrução?
\end{itemize}

    \item[IV --] \textbf{Espaço de Trabalho}

Acesse as \textbf{Definições} do \textbf{Espaço de Trabalho} “exemplo”.

Observe por 15 segundos esta página e responda a seguir.

\begin{itemize}
    \item Qual a finalidade dessa página para você? 
    \item Você consegue dizer quais os recursos estão disponíveis nessa página?
\end{itemize}

    \item[V --] \textbf{Gerenciamento das Turmas}
    
Acesse a página para \textbf{Gerenciar as Turmas} do “exemplo”.

Observe a página por 15 segundos e responda:

\begin{itemize}
    \item Qual a finalidade dessa página para você? 
    \item Você consegue dizer quais os recursos estão disponíveis nessa página?
\end{itemize}

Visualize os alunos de alguma das turmas.

Insira um novo aluno na turma.

Remova esse aluno da turma.

Retorne as \textbf{Definições} do \textbf{Espaço de Trabalho} “exemplo”.

\begin{itemize}
    \item Você teve alguma dificuldade em seguir alguma instrução até esse momento?
\end{itemize}
    
    \item[VI --] \textbf{Criar Tabela de Horários}

Crie uma \textbf{Tabela de Horário} para “exemplo”.

Visualize as opções no menu secundário da página.

Observe a página por 15 segundos e responda a seguir:

\begin{itemize}
    \item Qual a finalidade dessa página para você? 
    \item Você consegue dizer quais os recursos estão disponíveis nessa página?
\end{itemize}

Aumente a quantidade de horários de prova.

Defina o horário em que a primeira turma deverá realizar prova.

Defina o horário para as demais turmas de “exemplo” realizarem prova.

Observe a página por 10 segundos e responda a seguir:

\begin{itemize}
    \item Você consegue dizer quais os recursos estão disponíveis nessa página?
\end{itemize}

Gere um \textbf{Relatório} com os horários de prova que você determinou.

    \item[VII --] \textbf{Relatório}

Observe a página por 15 segundos e responda a seguir:

\begin{itemize}
    \item Qual a finalidade dessa página para você? 
    \item Você consegue dizer quais os recursos estão disponíveis nessa página?
\end{itemize}

Visualize os alunos que deverão realizar duas provas diferentes em um mesmo horário.

Utilize o \textbf{Robô} para encontrar uma nova tabela de horários de prova.

    \item[VIII --] \textbf{Robô}
    
Observe a página por 15 segundos e responda a seguir:

\begin{itemize}
    \item Existe alguma diferença para a página anterior? 
\end{itemize}

Salve o relatório do resultado como um \textbf{PDF}.

    \item[IX --] \textbf{Avaliação Geral}
    
Responda as perguntas a seguir para finalizar o teste.

\begin{itemize}
    \item Você utilizaria esse sistema para gerar grades de horários de prova?
    \item Entre 1 e 5, qual seria sua avaliação sobre sua utilização do sistema?
    \item Você recomendaria esse sistema para uma pessoa que precisasse criar grades de horários de prova?
    \item Você conhece ou já utilizou algum sistema similar a esse?
    \item Se você precisasse definir o sistema em uma palavra, qual seria?
\end{itemize}

Muito obrigado por participar deste teste de usabilidade.

\end{enumerate}

\end{apendicesenv}