\chapter{Considerações Finais}
\label{cap:conclusao}
% ---

% \section{Problema Geral}  \section{Problema Específico}

% JUSTIFICATIVA + TEMA
Ao produzir um sistema amigável utilizando a coloração de grafos de interseção para resolver o problema de programação de horário de exames em universidades, obtivemos exito em facilitar a obtenção do resultado de uma forma simplificada e organizada.

% \section{Objetivos Gerais}
O trabalho trouxe uma abordagem em que se obteve o melhor resultado possível considerando as turmas, seus alunos e a quantidade de horários de provas disponíveis na instituição. Sendo apresentado conceitualmente e de forma prática na aplicação proposta, a melhor alocação e suas possíveis consequências, permitindo dessa forma, que fossem realizadas as aplicações das provas de segunda chamada da melhor forma possível.

% \section{Objetivos Específicos}

Mesmo possuindo vasta quantidade e qualidade de referências e propostas de algoritmos eficientes que objetivassem solucionar o problema, não foi observado em nenhuma das pesquisas citadas uma aplicação voltada para o usuário final, considerando a possibilidade da utilização do “software” por um leigo em matemática ou computação.

No decorrer do trabalho, foi apresentado formalmente o problema de programação de horários de exames universitários com suas respectivas classificações e restrições, demonstrada a modelagem matemática para o problema e propondo um algoritmo capaz de obter o melhor resultado, além de apresentado um sistema que permite a realização da busca pela resolução de forma automática, contudo se mantendo amigável para qualquer usuário.

% \section{Hipótese / Estratégia}

A documentação do processo, partindo da identificação de um problema de natureza combinatória até o desenvolvimento de um produto amigável para um usuário final, por meio da utilização de abstrações, estruturas e modelos matemáticos, não é comum em trabalhos de conclusão de curso da instituição, contudo, visando uma maior disseminação dessas informações no âmbito da instituição e possibilitar a (re)utilização futura de partes deste trabalho em pesquisas de problemas de natureza similar, optamos seguir por este caminho.

% \section{Organização do Trabalho}

Buscando desenvolver um “software” fosse capaz de resolver o problema, mantendo sua simplicidade e sendo intuitivo ao máximo para os usuários finais, foi necessário que o problema estivesse bem estruturado, possibilitando a solução encontrada por meio da aplicação fosse completamente satisfatória tanto do ponto de vista de usabilidade, quanto do ponto de vista conceitual e matemático.

% RESULTADOS OBTIDOS

\section{Limitações da Pesquisa}

No decorrer da pesquisa, uma série de referências foram consultadas a fim de fundamentar de forma expressiva os métodos, ferramentas e expressões usadas no trabalho. No entanto devido ao excedente de informações relativas a diversos pontos, fez-se necessário que o trabalho adotasse uma postura mais didática e por vezes superficial sobre alguns temas.

Devido a opção de se realizar um trabalho mais didático e prático, considerando ser uma pesquisa de graduação tecnóloga de Análise de Sistemas, a parte matemática do trabalho carece de maiores detalhes e aprofundamento. Contudo as referências citadas ao longo do trabalho possuem pesquisas com grande enfoque matemático, sendo de certa forma complementar ao apresentado.

A modelagem UML do sistema carece de uma documentação mais extensa ao se comparar com as recomendações de autores do tema, contudo a escolha por realizar uma documentação mais simples foca-se na flexibilização para possibilitar a reutilização da modelagem em futuras implementações. Contudo a opção deixa a documentação mais distante do sistema a ser desenvolvido de fato.

Na parte da análise dos resultados obtidos e observados com a utilização do protótipo através do teste de usabilidade não foi realizado, devido as restrições impostas pelo distanciamento social causado pela pandemia de Covid-19 causada pelo vírus Sars-Cov2, popularmente conhecido como Coronavírus, contudo o registro do questionário e roteiro permitirá a aplicação do experimento e a consequente validação do protótipo em oportunidades futuras.


\section{Recomendações para Pesquisas Futuras}

Os problemas de programação de horários são vastos na literatura e permitem a resolução de problemas de naturezas diferentes, assim como os problemas similares citados, de alocação de bandas em dias de um festival, escalas de equipes em uma companhia aérea e a distribuição de recursos militares. Existem ainda muitos outros problemas não enunciados que podem ser modelados da mesma forma e por fim serem solucionados pelos mesmos métodos.

Estudos de problemas similares podem focar tanto em explorar as abordagens da solução do modelo matemático, através de métodos determinísticos ou não-determinísticos, ou na criação de sistemas personalizados voltados para resolução de uma determinada demanda.

Trabalhos nas áreas da Teoria dos Grafos e de resolução de problemas de natureza combinatória por muitas vezes focam-se apenas na modelagem e solução matemática ou algorítmica do modelo, e muitas vezes não produz um sistema voltado para um usuário real, é recomendada a implementação de sistemas que apliquem essas as soluções, hoje abstratas, para muitos problemas.

Observa-se também que existe a possibilidade de se expandir as variáveis consideradas no problema, de forma que sejam ponderadas questões como quantidade máxima de alunos a realizarem provas em um mesmo horário, quantidade de salas disponíveis simultaneamente para realização dos exames, quantidade de professores ou aplicadores disponíveis em cada horário, minimização da quantidade de provas de segunda chamada para um mesmo aluno, entre muitos outros que poderiam ser inclusos.

Ao se prosseguir com pesquisas e desenvolvimento, no âmbito principalmente da graduação, de ferramentas voltadas para temas similares ao apresentado nesse trabalho, cada vez mais serão populares os métodos de modelagem e resolução de problemas de natureza combinatória, que estão entre os mais complexos da computação e talvez os que resolvam o maior número de problemas cotidianos considerados complicados. Também será aumentada a quantidade de aplicações automatizadas para problemas cotidianos, beneficiando dessa forma toda a sociedade, além proporcionar aumento considerável do acervo de conhecimento científico em língua portuguesa sobre esses problemas.