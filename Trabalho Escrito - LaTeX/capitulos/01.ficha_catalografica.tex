
% Isto é um exemplo de Ficha Catalográfica, ou ``Dados internacionais de
% catalogação-na-publicação''. Você pode utilizar este modelo como referência.
% Porém, provavelmente a biblioteca da sua universidade lhe fornecerá um PDF
% com a ficha catalográfica definitiva após a defesa do trabalho. Quando estiver
% com o documento, salve-o como PDF no diretório do seu projeto e substitua todo
% o conteúdo de implementação deste arquivo pelo comando abaixo:
%
% \begin{fichacatalografica}
%     \includepdf{fig_ficha_catalografica.pdf}
% \end{fichacatalografica}

\begin{fichacatalografica}
	\sffamily
	\vspace*{\fill}					% Posição vertical
	\begin{center}					% Minipage Centralizado
	\fbox{\begin{minipage}[c][7,5cm]{12.5cm} 		% Largura
	\small
	
	C198m 
	\vspace{-0.3cm}
	\hspace{0.2cm}
	\parbox[t]{11cm}{
	%Sobrenome, Nome do autor
	Silva, Filipe Rodrigues Cardoso da
	
	\hspace{0.5cm} %\imprimirtitulo \imprimirautor. --
	%\imprimirlocal, \imprimirdata-
	Utilizando Coloração de Grafos de Interseção para Resolver o Problema de Programação de Horários de Exames em Universidades / Filipe Rodrigues Cardoso da Silva. – Rio de Janeiro: [s. n.], \imprimirdata.
	
	\hspace{0.5cm} \pageref{LastPage} p. 29 cm.\\
	
	\hspace{0.5cm}%  \imprimirorientadorRotulo~\imprimirorientador\\
	
	\hspace{0.5cm}
	%\parbox[t]{11.9cm}{
	Trabalho de Conclusão de Curso (Graduação em Tecnólogo
em Análise e Desenvolvimento de Sistemas) ~--~ Faculdade de Educação Tecnológica do Estado do Rio de Janeiro - FAETERJ-Rio, 
	\imprimirdata.
	%}
	\\
	
	\hspace{0.5cm}
	1. Coloração de Grafos. 
	2. Grafos de Interseção. 
	2. Programação de Horários de Exames em Universidades. 
	I. Silva, Filipe Rodrigues Cardoso da. 
	II. FAETERJ-Rio. 
	III. Utilizando Coloração de Grafos de Interseção para Resolver o Problema de Programação de Horários de Exames em Universidades  			
	}
	\begin{flushright}
	CDD 371.2
	\end{flushright}
	\end{minipage}}
	\end{center}
	
\end{fichacatalografica}
% ---